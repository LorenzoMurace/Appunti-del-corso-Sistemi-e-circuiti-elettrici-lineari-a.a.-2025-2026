\section{Resistore lineare}
\begin{dfn}[Resistore lineare]
	Componente circuitale in cui si "concentra" la proprietà fisica della resistenza elettrica. È più che un semplice modello del resistore fisico, in quanto può assumere le proprietà di resistenza di altre parti del circuito. 
\end{dfn}
\begin{legge}[Legge costitutiva del resistore lineare] \[ v(t) = R i(t)\]
	\[i(t) = G v(t)\]
	secondo la convenzione dell'utilizzatore. $R$ è detta \textsl{resistenza elettrica} ed è misurato in $\Omega$ (\textsl{Ohm}); $G =\frac{1}{R}$ \textsl{conduttanza elettrica}, misurata in $S$ (Siemens)
\end{legge}
Classificazione del componente:
\begin{itemize}
	\item Controllato in corrente e in tensione;
	\item Lineare ($R$ non dipende da $i(t)$; $G$ non dipende da $v(t)$);
	\item Senza memoria;
	\item Tempo-invariante ($R$ è costante);
	\item Passivo ($p_A(t)= Ri^2(t) = G v^2(t) \geq 0 \Rightarrow W_A(t) \geq 0$;
\end{itemize}
\subsection{Resistori in serie e partitore di tensione}
Resistori in serie:
\[LKT_{\Gamma}: v= \sum_k v_k= (\sum_k R_k) i = R_{eq} i\]
con $R_{eq} = \sum_k R_k$. Posso dunque semplificare il circuito considerando la serie come un unico resistore con resistenza $R_{eq}$.
\begin{dfn}[Partitore di tensione]
	Serie di resistori. Ci interessiamo alla caduta di tensione dopo ciascun resistore.\\
	Detta $v$ la tensione applicata alla serie, la caduta di tensione sul resistore $k$-esimo è \[v_k = \frac{R_k}{R_{eq}}v\]
\end{dfn}
\subsection{Resistori in parallelo e partitore di corrente}
Resistori in parallelo:
\[LKC_A: i = \sum_k i_k = \sum_k G_kv = (\sum_k G_k) v = G_{eq} v\]
con $G_{eq} = \sum_k G_k = \frac{1}{R_{eq}}$, da cui $R_{eq} = \frac{1}{\sum_k \frac{1}{R_k}}$. Posso dunque semplificare il circuito considerando la serie come un unico resistore con resistenza $R_{eq}$.
\begin{dfn}[Partitore di corrente]
	Parallelo di resistori. Ci interessiamo alla corrente che scorre in ciascun resistore.\\
	Detta $i$ la corrente che scorre nel parallelo, la corrente che scorre nel resistore $k$-esimo è \[i_k = \frac{G_k}{G_{eq}}i = \frac{R_{eq}}{R_k} i \]
\end{dfn}
\subsection{Cortocircuito}
Nel caso in cui uno dei rami del parallelo abbia resistenza nulla, tutta la corrente passa in quel ramo, e la tensione ai due capi risulta nulla. Definiamo un ramo così caratterizzato \textsl{componente cortocircuito}.
\begin{dfn}[Componente cortocircuito]
	Tratto di conduttore ideale
\end{dfn}
\begin{legge} [Legge costitutiva del cortocircuito]
	\[v(t) = 0 \quad \forall i(t)\]
\end{legge}
Osserviamo che questo è un componente inerte ($p_A(t)= 0 \quad \forall i(t)$;
Un cortocircuito reale ha una resistenza molto piccola ma finita, dunque vi può circolare un'elevata corrente, che può essere dissipata in calore al punto da generare, ad esempio, incendi.

\subsection{Circuito aperto}
Simmetricamente, nel caso in cui un ramo abbia resistenza infinita, tutta la corrente passa nel ramo opposto, e tale si comporta come un circuito aperto. 
\begin{dfn}[Componente circuito aperto]
	Tratto con resistenza infinita.
\end{dfn}
\begin{legge} [Legge costitutiva del cortocircuito]
	\[i(t) = 0 \quad \forall v(t)\]
\end{legge}
Osserviamo che anche questo è un componente inerte ($p_A(t)= 0 \quad \forall v(t)$.

\subsection{Trasformazioni triangolo-stella nella connessione di resistori}
\begin{figure}[H]
	\centering
	\begin{subfigure}[b]{0.45\textwidth}
		\centering
		\includegraphics[width=\linewidth]{triang}
	\end{subfigure}
	\hfill
	\begin{subfigure}[b]{0.45\textwidth}
		\centering
		\includegraphics[width=\linewidth]{stella}
	\end{subfigure}
	\caption{Configurazioni di resistori a triangolo (a sinistra) e a stella (a destra).}
	\label{fig:triang_stella}
\end{figure}
Per ottenere la \textbf{\textsl{stella equivalente al triangolo}}:
\begin{itemize}
	\item [1] Aggiungere un centro-stella;
	\item [2] Disegnare $R_A$, $R_B$, $R_C$
	\item [3] Si dimostra che: 
	\begin{equation}\tag{trasformazioni triangolo-stella}
		\label{eq:triangolo-stella}
		\begin{split}
			R_A = \frac{R_{AB}R_{AC}}{R_{AB} + R_{BC} + R_{AC}}\\
			R_B = \frac{R_{AC}R_{BC}}{R_{AB} + R_{BC} + R_{AC}}\\
			R_A = \frac{R_{AB}R_{AC}}{R_{AB} + R_{BC} + R_{AC}}
		\end{split}
	\end{equation}
\end{itemize}
Nel caso particolare $R_{AB}=R_{BC}=R_{AC}= R_{\Delta}$, $R_A = R_B = R_C = R_{\Delta}/3$.\\
\\

Al contrario, per ottenere il \textbf{\textsl{triangolo equivalente alla stella}}:
\begin{itemize}
	\item [1] Rimuovere centro-stella;
	\item [2] Disegnare $R_{AB}$, $R_{BC}$, $R_{AC}$
	\item [3] Si dimostra che: 
	\begin{equation}\tag{trasformazioni stella-triangolo}
		\label{eq:stella-triangolo}
		\begin{split}
			R_{AB} = \frac{R_{A}R_{B} + R_{B}R_{C} + R_{A}R_{C}}{R_C}\\
			R_{BC} = \frac{R_{A}R_{B} + R_{B}R_{C} + R_{A}R_{C}}{R_A}\\
			R_{AC}=  \frac{R_{A}R_{B} + R_{B}R_{C} + R_{A}R_{C}}{R_B}
		\end{split}
	\end{equation}
\end{itemize}
Nel caso particolare $R_{A}=R_{B}=R_{C}= R_{stella}$, $R_{AB} = R_{BC} = R_{AC} = 3R_{stella}$.\\
\\
Per un esercizio sul tema, si faccia riferimento all'\cref{esercizio:3}. L'esercizio è riportato in seguito perché comprende un generatore, componente che sarà trattato nel prossimo capitolo. 


\section{Generatori indipendenti di tensione e di corrente}
\subsection{Generatori ideali}
\begin{dfn}[Generatore indipendente di tensione]
	Nella notazione U.S., il simbolo è un cerchio con un segno + e uno - che ne indicano la polarità. Chiamiamo $V_g$ la fem del generatore. Nella notazione UE e ITA, il simbolo è un cerchio contenente una freccia che indica nel verso della polarizzazione. Per questo componente si utilizza la convenzione dell'utilizzatore.
\end{dfn}
\begin{legge}[Legge costitutiva del generatore di tensione]
	\[v(t) = V_g \quad \forall i(t)\]
\end{legge}
Classificazione:
\begin{itemize}
	\item Controllato in corrente (non possibile scriverlo in una forma controllata in tensione)
	\item Non lineare
	\item Senza memoria
	\item Attivo (Comportamento energetico: $P_E(t) = V_g i(t)$, per ogni caso; può anche assorbire energia oltre a erogarne)
\end{itemize}

\begin{dfn}[Generatore indipendente di corrente]
	Nella notazione U.S., il simbolo è un cerchio con una freccia verso l'alto che indica il verso di riferimento della corrente generata. Chiamiamo $I_g$ la corrente generata dal generatore. Nella notazione UE e ITA, il simbolo è un cerchio contenente un trattino. Per questo componente si utilizza la convenzione dell'utilizzatore.
\end{dfn}
\begin{legge}[Legge costitutiva del generatore di corrente]
	\[i(t) = I_g \quad \forall v(t)\]
\end{legge}
Classificazione:
\begin{itemize}
	\item Controllato in tensione (non possibile scriverlo in una forma controllata in corrente)
	\item Non lineare
	\item Senza memoria
	\item Attivo (Comportamento energetico: $P_E(t) = v(t)I_g$, per ogni caso; può anche assorbire energia oltre a erogarne)
\end{itemize}
Per giustificare il fatto che dei generatori possano assorbire energie, si pensi a componenti reali modellizzati da questi componenti, quali le batterie delle auto elettriche, che possono assorbire energia dalle frenate.

Sono circuiti impossibili:
\begin{itemize}
	\item Generatore di tensione cortocircuitato.
	\item Generatore di corrente in serie a un circuito aperto.
\end{itemize}

\begin{ex}
	\label{ex:1}
	In \cref{fig:esercizio1}, siano $R_1 = 1 \Omega$, $R_2 = 6 \Omega$, $R_3 = 4 \Omega$, $R_4 = 2 \Omega$, $R_5 = 1 \Omega$, $I_g = 1A$.\\
	\begin{figure}[h]
		\centering
		\includegraphics[width=0.7\linewidth]{Esercizio1}
		\caption{}
		\label{fig:esercizio1}
	\end{figure}
	Verifichiamo il teorema di conservazione della potenza.
	Si semplifica il circuito, considerando che:
	\begin{enumerate}
		\item $R_5$ è in parallelo al cortocircuito (\cref{fig:ex11})
		\begin{figure}[h!]
			\centering
			\includegraphics[width=0.5\linewidth]{ex1_3}
			\caption{}
			\label{fig:ex11}
		\end{figure}	
		
		\item $R_3$ e $R_4$ sono in serie, con resistenza equivalente $R_{eq1} = 6 \Omega$ (\cref{fig:ex12})
			\begin{figure}[H]
			\centering
			\includegraphics[width=0.5\linewidth]{ex1_2}
			\caption{}
			\label{fig:ex12}
		\end{figure}
		
		\item $R_2$ è in parallelo a  $R_{eq_1}$, con $R_{eq_2}= 3 \Omega $ (\cref{fig:ex13})
			\begin{figure}[h!]
			\centering
			\includegraphics[width=0.5\linewidth]{ex1_1}
			\caption{}
			\label{fig:ex13}
		\end{figure}
		
		\item $R_1$ è in serie a $R_{eq_2}$ con $R_{eq}= 4 \Omega$
	\end{enumerate}
	
	\[P_{A, R_{eq}} = R_{eq} i_{eq}^2 =  4 \Omega \cdot \ (1A)^2 = 4\  W\]
	\[P_{E, I_g} = V_g \cdot I_g = v_{R_{eq}} I_g = R_{eq} i_{R_eq} I_g = 4\ W = P_{A, R_{eq}}\]
	Calcoliamo anche la potenza assorbita da ciascun resistore. Procediamo a ritroso nella semplificazione del circuito: 
	\begin{itemize}
		\item dalla \cref{fig:ex13}, $P_{A,R_1}= R_1 i_{R_1}^2 = 1\ W$;
		\item dalla \cref{fig:ex12}, $i_{R_2} = \frac{E_{eq1}}{R_2 + R_{eq1}}= 0,5\ A$, dunque $P_{A,R_2}= R_2 i_{R_2}^2 = 1,5\ W$;
		\item dalla \cref{fig:ex11}, $i_{R_3} = i_{R_1} - i_{R_2}= 0,5\ A$, da cui $P_{A,R_3}= 1\ W$ e $P_{A,R_4}= 1\ W$. 
	\end{itemize}
	Risulta comunque che:
	\[P_{E,g}- \sum_k P_{A,R_k} = 0\]
\end{ex}

\subsection{Generatori reali}
\begin{figure}[H]
	\centering
	\includegraphics[width=1.0\linewidth]{genReali}
	\caption{Schema della rappresentazione di generatori reali tramite componenti ideali.}
	\label{fig:genreali}
\end{figure}
Per un generatore di tensione, un generatore reale può essere rappresentata come un generatore reale in serie a una resistenza interna (detta \textsl{resistenza serie}, $R_g$) (A sinistra in \cref{fig:genreali}).
\[LKT:\ v = V_g -v_{R_g} = V_g - R_g i \]
$v_{R_g}$ è detta \textsl{caduta di tensione} sulla resistenza interna. A parità di corrente erogata, maggiore è la resistenza interna, minore la tensione effettivamente generata dal generatore reale. Per questa ragione, in ambito circuitale, la fem è definita come la tensione "a vuoto" di un generatore reale, ed è dunque distinta dalla tensione effettivamente generata dal generatore in attività.\\
\\
Per un generatore di corrente, un generatore reali può essere rappresentato come un generatore reale in parallelo a una resistenza interna (detta \textsl{resistenza parallelo}. $R_g$) (A destra in \cref{fig:genreali}).
\[LKC:\ i = I_g - i_{R_g} = I_g - \frac{v}{R_g} \]
A pari tensione, all'aumentare della resistenza parallelo, aumenta la corrente effettivamente erogata, fino al valore di $I_g$.\\
\subsection{Generatori pilotati}
\begin{dfn}
	Un generatore la cui variabile (tensione/corrente) imposta dal generatore viene pilotata da un'altra tensione/corrente del circuito.
\end{dfn}
\begin{description}
	
	\item [Generatore di tensione pilotato in tensione] $v(t) = \alpha v_k(t) \quad \forall i(t)$, con $\alpha$ detto \textsl{guadagno in tensione}.
	\begin{figure}[H]
		\centering
		\includegraphics[width=0.5\linewidth]{genvpilv}
		\caption{Esempio di circuito contenente un generatore di tensione pilotato in tensione}
		\label{fig:genvpilv}
	\end{figure}
	
	\item [Generatore di tensione pilotato in corrente] $v(t) = r i_k(t) \quad \forall i(t)$, con $r$ detto \textsl{resistenza di trasferimento}
	\begin{figure}[H]
		\centering
		\includegraphics[width=0.3\linewidth]{genvpili}
		\caption{Simbologia per un generatore di tensione pilotato in corrente}
		\label{fig:genvpili}
	\end{figure}
	
	\item [Generatore di corrente pilotato in corrente] $i(t) = \beta i_k(t) \quad \forall v(t)$, con $\beta$ detto \textsl{guadagno in corrente}. 
		\begin{figure}[H]
		\centering
		\includegraphics[width=0.3\linewidth]{genipili}
		\caption{Simbologia per un generatore di corrente pilotato in corrente}
		\label{fig:genipili}
	\end{figure}
	
	\item [Generatore di corrente pilotato in tensione] $i(t) = g v_k(t) \quad \forall v(t)$, con $g$ detto \textsl{conduttanza di trasferimento}.
	\begin{figure}[H]
		\centering
		\includegraphics[width=0.3\linewidth]{genipilv}
		\caption{Simbologia per un generatore di corrente pilotato in tensione}
		\label{fig:genipilv}
	\end{figure}
\end{description}
Classificazione:
\begin{itemize}
	\item senza memoria
	\item non lineari
	\item attivi
\end{itemize}

\begin{ex}
	\begin{figure}[H]
		\centering
		\includegraphics[width=0.5\linewidth]{expil}
		\caption{}
		\label{fig:expil}
	\end{figure}
	Siano: $R_1 = 1\ \Omega$, $R_2 = 5 \ Omega$, $V_g = 45V$, $\alpha = 3$. Calcolare la corrente circolante nel circuito.
	\begin{itemize}
		\item scrivere una LK che includa la variabile pilotata del generatore.
		\[LKT_I = V_g - v_{R_1} - v_{R_2} +\alpha v_{R_1}\]
		\item Eventualmente, sostituire le tensioni/correnti nella LK con leggi costitutive.
		\[LKT_I = V_g - R_1i - R_2i +\alpha R_1i\]
		Da cui:
		\[i=\frac{V_g}{R_1 (1 - \alpha) + R_2} = 15 \ A\]
	\end{itemize}
\end{ex}

\begin{ex}
	\label{esercizio:3}
	Calcolare $P_{E,g}$ nel circuito in \cref{fig:esercizio2}.
	\begin{figure}[H]
		\centering
		\includegraphics[width=0.7\linewidth]{Esercizio2}
		\caption{}
		\label{fig:esercizio2}
	\end{figure}
	Procediamo alla trasformazione stella-triangolo dei resistori 2, 3 e 4. Otteniamo il circuito in \cref{fig:ex2trasf}.
	\begin{figure}[H]
		\centering
		\includegraphics[width=0.7\linewidth]{ex2trasf}
		\caption{}
		\label{fig:ex2trasf}
	\end{figure}
	Applicando le \ref{eq:triangolo-stella},$R_{AB} = 54\ \Omega $, $R_{BC} = 13,5\ \Omega$, $R_{AC} = 54\ \Omega$.
	Procediamo con la semplificazione serie-parallelo:
	\begin{itemize}
		\item $R_1$ è in parallelo a $R_{AB}$, con $R_{eq1}=\frac{27}{5}\ \Omega$;
		\item $R_5$ è in parallelo a $R_{BC}$, con $R_{eq2}=\frac{27}{5}\ \Omega$;
		\item $R_{eq1}$ e $R_{eq2}$ sono in serie con $R_{eq3} = \frac{54}{5}\ \Omega$;
		\item $R_{AC}$ è in parallelo a $R_{eq3}$, con $R_{eq}= 9\ \Omega$.
	\end{itemize}
	Risulta dunque che $P_{E_g} = P_{A,R_{eq}} = \frac{V_g^2}{R_{eq}} = 144 W$.\\
	\\
	Analogamente, si può risolvere l'esercizio applicando una trasformazione triangolo-stella, ottenendo il circuito in \cref{fig:ex2trasfalt}.
	\begin{figure}[H]
		\centering
		\includegraphics[width=0.7\linewidth]{ex2trasfalt}
		\caption{}
		\label{fig:ex2trasfalt}
	\end{figure}
	Applicando le \ref{eq:stella-triangolo},$R_{A} = 4\ \Omega $, $R_{B} = 1\ \Omega$, $R_{O} = 4\ \Omega$.
	Procediamo con la semplificazione serie-parallelo:
	\begin{itemize}
		\item $R_O$ è in serie con $R_{4}$, con $R_{eq1}=10 \Omega$
		\item $R_B$ è in serie con $R_{5}$ $R_{eq2}=10 \Omega$;
		\item $R_{eq1}$ è in parallelo a $R_{eq2}$, con $R_{eq3}=5\ \Omega$;
		\item $R_{eq1}$ e $R_{eq2}$ sono in serie con $R_{eq3} = \frac{54}{5}\ \Omega$;
		\item $R_{A}$ è in serie a $R_{eq3}$, con $R_{eq}= 9\ \Omega$.
	\end{itemize}
	Risulta quindi, $P_{E_g} = P_{A,R_{eq}} = \frac{V_g^2}{R_{eq}} = 144 W$, coerentemente con quanto risultante dalla prima risoluzione.
\end{ex}
