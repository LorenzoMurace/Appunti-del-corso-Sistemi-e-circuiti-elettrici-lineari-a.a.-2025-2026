\section{Introduzione ai circuiti elettrici a parametri concentrati}
\subsection{Definizioni fondamentali}
Si distinguerà d'ora in poi tra:
\begin{description}
	\item [\textsl{circuito fisico}]: ha luogo nella realtà ed è costituito da componenti materiali (es. un filo in cui circola corrente, la stessa per ogni sezione, in quanto è un tubo di flusso di $\vec J_t$). Risolverlo significa risolvere le equazioni di Maxwell nella materia che costituiscono un sistema di equazioni differenziali alle derivate parziali $f(x,y,z,t)$. Si ottiene come risultato una soluzione esatta, analitica per i valori di tutti i campi in ogni posizione nello spazio: problema $3D$ (tre coordinate spaziali e una temporale)
	\item[\textsl{circuito a parametri concentrati}]: Rappresentazione approssimata di un circuito fisico, mediante \textsl{componenti (circuitali)} e \textsl{collegamenti ideali}. Utile a una risoluzione approssimata ma comunque affidabile e più rapida del sistema. Rinunciando alla descrizione accurata (in termini di coordinate spaziali), risolvere il circuito significa risolvere un sistema di equazioni differenziali ordinarie: problema $0D$ (solo coordinata temporale).\\
	Si fonda su tre ipotesi:
	\begin{itemize}
		\item i collegamenti sono ideali ($\sigma = \infty\ \iff \rho = 0$)
		\item lo spazio esterno ai collegamenti e ai componenti è perfettamente isolante ($\sigma = 0\ \iff \rho = \infty$)
		\item nello spazio esterno, (1) $\frac{\partial \vec B}{\partial t}\approx 0$, cosicché $\vec E$ sia conservativo e $\vec E = \nabla \varphi$ dove $\phi$ è il potenziale elettrico, misurato in $V$ e (2) $\frac{\partial \vec D}{\partial t}\approx 0$, cosicché $\nabla \times \vec H = \vec J$ e dunque $\vec J$ è un campo solenoidale ($\nabla \cdot \vec J = 0$). Da queste due condizioni derivano le Leggi di Kirchhoff.
	\end{itemize}
\end{description}
\subsubsection{Parti del circuito}
\begin{dfn}[componente]
	Sottosistema delimitato da una \textsl{superficie limite} che può essere attraversata da correnti sono in corrispondenza dei \textsl{terminali}, ossia le regioni di contatto con i collegamenti esterni.\\
	Un componente con 2 terminali è detto \textsl{dipolo}; uno con 3, \textsl{tripolo}, uno con n, \textsl{n-polo}.
\end{dfn}
\begin{dfn}[Nodo (polo)] Un nodo (o polo) si intende indistintamente una delle due seguenti configurazioni:
	\begin{enumerate}
		\item terminale isolato
		\item insieme di più terminali connessi da collegamenti ideali.
	\end{enumerate}
\end{dfn}
\begin{dfn}[Ramo]
	Tratto del circuito che unisce due nodi adiacenti.
\end{dfn}
\begin{legge}[Principio di invarianza topologica]
	Osserviamo che alcuni fatti soltanto alcuni fatti sono rilevanti per il comportamento del circuito.
	\begin{description}
		\item[Influenzano il comportamento del circuito:] \leavevmode
		\begin{itemize}
			\item la tipologia dei componenti
			\item la topologia del circuito
		\end{itemize}
		\item[Non influenzano il comportamento del circuito:] \leavevmode
		\begin{itemize}
			\item la posizione dei componenti nello spazio (infatti la sua rappresentazione non corrisponde a posizioni spaziali nella realtà)
			\item la forma dei collegamenti (che infatti sono ideali)
		\end{itemize}
		
	\end{description}
\end{legge}
\begin{dfn}[Maglia]
	Percorso chiuso formato da rami del circuito che attraversa una sequenza chiusa di nodi, senza passare più di una volta per lo stesso nodo.
\end{dfn}
\subsubsection{Grandezze}
\begin{dfn}[Corrente]
	La consideriamo dovuta a un moto di cariche positive (+), per convenzione.\\
	\[i=\int_S \vec J \cdot d\vec S = \frac{dQ}{dt}\]
	con $S$ generica perché il circuito è un tubo di flusso per $J$ campo solenoidale (per le ipotesi sopra).\\ 
	Ne risulta che:
	\[\Delta Q = \int_t^{t+\Delta t} i(t)dt\]
	È necessario definire un \textsl{verso di riferimento (VDR)} per effettuare il calcolo di una corrente $i$ tra due punti, in analogia con la necessità di assegnare un verso alla normale della superficie per calcolare il flusso. Tale verso è (1) arbitrario e (2) costante nel tempo. Effettuati i calcoli, è possibile che il \textsl{verso effettivo} risulti concorde ($i>0$) oppure discorde ($i<0$) con quello di riferimento.
\end{dfn}
\begin{dfn}[Tensione]\label{dfn:tensione}
	Variazione di energia per unità di carica quando una carica puntiforme si sposta da un A a B.
	\[v_{AB}=\int_A^B \vec E \cdot d \vec l = - \int_A^B \nabla \varphi \cdot d \vec l = - (\varphi(B) - \varphi(A))= \varphi_A - \varphi_B\]
	Tensione e differenza di potenziale in generale non sono la stessa cosa. Coincidono soltanto nel caso in cui $\nabla \times \vec E = 0$, ossia se $\vec E$ è conservativo.\\
	Per calcolare la tensione tra due punti, è necessario determinare un verso di riferimento, arbitrario e costante. Se $v_{AB}>0$, una carica che si sposta da $A$ a $B$ perde energia potenziale elettrica. Viceversa, $v_{AB}<0$, una carica che si sposta da $A$ a $B$ perde energia.
\end{dfn}
\subsection{Leggi di Kirchhoff - Leggi topologiche}
\begin{legge}[Legge di Kirchhoff delle correnti - LKC]
	La somma algebrica delle correnti attraverso qualsiasi superficie $S$ chiusa esterna alle superfici limite dei componenti è nulla.\\
	Per esempio, si consideri una superficie che taglia tre rami e include un nodo ($S_1$ in \cref{fig:lkc}). Possibile considerare anche una superficie che include più nodi. Oppure ancora, una superficie che circondi sostanzialmente soltanto un nodo ($S_2$ in \cref{fig:lkc}): in questo caso prende il nome di \textsl{legge nodale}.
	\begin{figure}[h]
		\centering
		\includegraphics[width=0.8\linewidth]{LKC}
		\caption{}
		\label{fig:lkc}
	\end{figure}
\end{legge}

Indaghiamo il collegamento tra la LKC e la Teoria dei campi. Considero un circuito  e una superficie che circondi un nodo tra più rami. Risulta (considerando il riferimento di \cref{fig:lkc}, e indicando le superfici di sezione del filo con apici): \[\oint_{S_2} \vec J_t \cdot d \vec S = \oint_{S_2} \vec J \cdot d \vec S = \int_{S_2'} \vec J \cdot d\vec S + \int_{S_2''} \vec J \cdot d\vec - \int_{S_2'''} \vec J \cdot d\vec S = i_4 + i_5 - i_6.\] Dunque la LKC è una conseguenza del carattere solenoidale di $\vec J$ (fuori dalle superfici limite dei componenti).\\
Conseguenze della LKC:
\begin{description}
	\item[corrente di ramo]. Un bipolo in generale avrebbe una corrente entrante e una uscente distinte. Ma per la LKC la loro somma algebrica deve essere nulla, dunque devono essere uguali.
	\item [corrente in un ramo aperto (circuito aperto)]. La corrente in un ramo aperto può soltanto essere nulla, altrimenti si avrebbe una corrente entrante nel nodo terminale e nessuna corrente uscente.
\end{description}

\begin{legge}[Legge di Kirchhoff delle tensioni - LKT]
	La somma algebrica delle tensioni lungo una curva chiusa esterna $\Gamma$ alle superfici limite dei componenti passante per una sequenza di nodi è nulla.\\
	Per esempio, considerando la \cref{fig:lkt}, assegnato verso di riferimento a $\Gamma_1$ e alle tensioni tra i nodi. Un caso particolare si ottiene considerando la curva $\Gamma_2$, coincidente con una maglia del circuito.
	
	\begin{figure}[h]
		\centering
		\includegraphics[width=0.8\linewidth]{LKT}
		\caption{}
		\label{fig:lkt}
	\end{figure}
	
	
\end{legge}
Indaghiamo il collegamento tra la LKT e la Teoria dei campi.
 \[\oint_{\Gamma_1} \vec E \cdot d \vec l = -\frac{d}{dt} \Phi_{B,\Gamma_1} = 0 = \sum_i \int_{A_i}^{B_i} \vec E \cdot d\vec l = \sum_i v_{A_i B_i}\]
\subsection{Potenza ed energia}
\begin{dfn}[Potenza per un bipolo]
	Consideriamo un bipolo generico sottoposto a una generica $v(t)$, attraversato da una $i(t)$. Definiamo la potenza istantanea (assorbita o erogata dal componente) $p(t) = \frac{dW(t)}{dt} = \frac{dW(t)}{dQ} \frac{dQ}{dt} = v(t)i(t)$.\\
	In circuiti in regime stazionario (corrente continua), $p(t) = P$ è la potenza, costante nel tempo.
\end{dfn}
Convenzioni nei versi di riferimento (calcolo potenza): 
\begin{description}
	\item[convenzione dell'utilizzatore] VDR i entrante nel polo (+) (definito in base a $v$): cariche da (+) a (-), quindi perdono energia (potenziale). L'energia (potenza) viene assorbita (dissipata) dal componente, dunque detto \textsl{utilizzatore}. $vi = P_A$ Potenza assorbita (se negativa, il componente eroga energia).
	\item[convenzione del generatore] VDR i uscente dal polo (+): cariche da (-) a (+), quindi guadagnano energia. L'energia (potenza) viene erogata dal componente. $vi = P_E$ Potenza erogata (se negativa, il componente assorbe energia).
\end{description}
Enunciamo ora un teorema fondato sulla validità delle Leggi di Kirchhoff.
\begin{thr}[Teorema di conservazione della potenza istantanea]
	Per ogni circuito elettrico, la somma algebrica delle potenze erogate è uguale alla somma algebrica delle potenze istantanee assorbite.\\
	Se $n$ sono componenti eroganti e $m$ sono utilizzatori.
	\[\sum_{i=1}^{n} P_{E,i}(t) = \sum_{j=1}^{m} P_{A,j}(t)\]
	Nella L5 è proposto un esempio di verifica del teorema.
\end{thr}
Implicazioni del teorema di conservazione della potenza istantanea. Siccome è valido anche nei circuiti reali. Ne vediamo alcuni esempi per la rete elettrica nazionale. Su dati.terna.it è possibile trovare dati sul fabbisogno nazionale nel tempo. L'andamento del fabbisogno è molto vario nel tempo (giorni, ore, mesi). Le fonti sono varie perché è necessario avere una adeguata adattabilità al fabbisogno, e l'accumulo di energia non è facile ed esteso.  L'idroelettrico (programmabile) è ormai al massimo, perché gli specchi d'acqua sfruttabili sono quasi al limite. Con quelle rinnovabili non programmabili è difficile ottenere di più in maniera affidabile.\\
Confronto tra i profili di produzione francese (con molto nucleare) e italiano (con più rinnovabili ma anche più gas).\\
Mappa dei consumi come indice di produttività della regione\\
Interventi sulla rete elettrica al 2030 e 2034: molti collegamenti. Obiettivo: renderla più "magliata" per consentire continuità elettrica. L'Italia è avanzata come paese industriale proprio grazie all'affidabilità della rete elettrica.\\
Crisi elettrica in Texas nel Febbraio 2021. Il Texas, per pagare meno tasse federali non è connesso alla rete elettrica federale (esistono una rete Western e una Eastern, con poche connessioni perché in mezzo ci sono le Montagne Rocciose) (dunque ha una rete propria, connessa con Messico e altri, ma molto più fragile: per questo è molto più esposto). Vedi immagini di Houston prima e dopo gli eventi (wikipedia). Le elevate spese italiane per l'interconnessione interna e con altri paesi ha grandi importanza per l'affidabilità della rete.\\
Ipotizzato lo stoccaggio dell'energia, per aumentare la penetrazione delle fonti di energia non programmabili. Per piccole comunità elettriche può funzionare (UPS). Per potenze molto elevate, sopperire alla non-programmabilità con le attuali batterie non è possibile, per costi e disponibilità di materiali. Tra le direzioni di sperimentazione, superconduttori e riciclo di batterie dall'automotive. Si sente di alcuni paesi che ottengono l'80\% della loro energia dal fotovoltaico, ma si tratta sempre di paesi piccoli: impossibile per fabbisogni più elevati.\\
\\
\begin{dfn}[Energia assorbita ed erogata]
	\[W(t) = \int_t^{t+\Delta t} p(t) dt = \int_t^{t+\Delta t} v(t) i(t)dt \]
	Secondo la convenzione già illustrata per la potenza, si parla di energia assorbita $W_A$ ed energia erogata $W_E$.
	Se $p(t)=P$, $W(\Delta t) = P \Delta t$.\\
	Unità di misura comuni, oltre ai $J$, sono i $Wh$ e il suo multiplo $kWh$.
\end{dfn}

\subsection{Leggi costitutive - Introduzione e classificazione}
Mentre le leggi di Kirchhoff sono \textsl{leggi topologiche}, studiamo ora le leggi che caratterizzano il comportamento di particolari componenti del circuito: le \textsl{leggi costitutive}, nella forma $f(v(t), i(t))=0$.
Classificazione dei componenti:
\begin{description}
	\item [In base alla forma della legge costitutiva (LC)]\leavevmode
	\begin{description}
		\item[Senza memoria (anche detti resistivi, algebrici, adinamici)] LC algebrica. Es. $v(t)=\alpha i(t) + \gamma$.
		\item [Con memoria (dinamici)] LC integro-differenziale. Es. $v(t)= \varepsilon \int_{-\infty}^t i(t)dt$
	\end{description}
	\item [In base alla linearità] lineari o non lineari.\leavevmode
	\item [In base alla tempo-varianza]\leavevmode
	\begin{description}
		\item[Tempo-invarianti] La dipendenza temporale non è esplicita ($t$ compare soltanto come argomento di $v(t), i(t)$)
		\item[Tempo-varianti] La dipendenza temporale è esplicita. Tutti i componenti lo sono, ma alcuni si possono approssimare a tempo-invarianti.
	\end{description}
	\item [In base ad attività o passività: ]\leavevmode
	\begin{description}
		\item[Passivi:] $W_A (t) =\int_{-\infty}^t p_A(t) dt \geq 0 \quad \forall t$. Il componente può anche erogare energia, ciò che lo definisce è che l'energia totale assorbita sia sempre positiva (può erogare soltanto energia assorbita, non ne genera).
		\item[Attivi: ] $\exists t\ | \quad W_A (t) <0 $.
	\end{description}
	\item [In base alla potenza scambiata: ]\leavevmode
	\begin{description}
		\item [Dissipativi:] $p_A (t) \geq 0 \quad \forall t$
		\item [Inerti: ] $p_A(t) = 0 \quad \forall t$
	\end{description}
	\item [In base al tipo di controllo] \leavevmode
	\begin{description}
		\item[controllati in tensione] Nella LC, $v(t)$ è la variabile indipendente
		\item [controllati in corrente] Nella LC, $i(t)$ è la variabile indipendente.
	\end{description}
\end{description}
\subsection {Collegamenti di bipoli in serie e parallelo}
\begin{dfn}[Serie]
	(1) Due bipoli sono collegati in serie se condividono un nodo a cui non afferiscono altri rami. Si osserva che in una configurazione del genere, i componenti in serie sono attraversati dalla stessa corrente; pertanto si può dare una definizione più generale di "collegamento in serie" (la 2).
	(2)Due o più bipoli sono in serie se sono attraversati dalla stessa corrente.
\end{dfn}
\begin{dfn}[Parallelo]
	(1) Due o più bipoli sono collegati in parallelo se sono connessi alla stessa coppia di nodi. Si osserva che in una configurazione del genere, i componenti in parallelo sono sottoposti alla medesima tensione; pertanto si può dare un'ulteriore definizione di "collegamento in parallelo" (la 2).
	(2) Due o più bipoli sono in parallelo se sono sottoposti alla stessa tensione.
\end{dfn}