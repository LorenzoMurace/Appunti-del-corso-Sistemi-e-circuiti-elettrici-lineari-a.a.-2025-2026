\section{Componenti dinamici}
\subsection{Condensatore ideale}
\begin{dfn}[Condensatore ideale]
	Componente di un circuito a parametri concentrati in cui si "concentra" la proprietà della \textsl{capacità} ($C$).
	\begin{figure}[H]
		\centering
		\includegraphics[width=0.8\linewidth]{condensatore}
		\caption{Rappresentazione schematizzata e simbolo in un circuito a parametri concentrati di un condensatore ideale}
		\label{fig:condensatore}
	\end{figure}
	\[q(t) = C v(t)\]
	\[\frac{d}{dt}q(t) = C \frac{d}{dt} v(t)\]
\end{dfn}
\begin{legge}[Legge costitutiva del condensatore ideale]
	\[i(t) = C \frac{d}{dt} v(t)\]
	\[v(t) = \frac{1}{C} \int_{-\infty}^t i(t)dt\]
	Classificazione:
	\begin{itemize}
		\item lineare 
		\item controllato in tensione o in corrente 
		\item con memoria (dinamico)
		\item se $v(t) = cost$, si comporta come un circuito aperto.
		\item componente passivo \[W_A(t) = C \int_{-\infty}^t v(t) \frac{dv(t)}{dt} = C \int_0^{v(t)} v dv = \frac{1}{2}Cv^2(t) \geq 0\] 
		L'energia immagazzinata nel condensatore è dovuta al campo elettrico. Si dice dunque che $v(t)$ è la variabile di stato di un condensatore, perchè, da sola, ne determina lo stato (Energia).
	\end{itemize}
\end{legge}
Osserviamo la proprietà di continuità di $v(t)$.\\
Se $v(t)$ è discontinua al tempo $t= \tau$
\[v(\tau ^-) \neq v(\tau ^+)\]
\[\Rightarrow i(\tau) = C \left. \frac{d v(t)}{dt} \right \vert_{\tau} = \infty\]
Perciò varrebbe l'assurdo \[P_A(\tau) = v(\tau) i(\tau) = \infty \]
Se ne conclude che non è possibile avere una tensione discontinua ai capi di un condensatore. Ossia $v(\tau^-) = v(\tau^+)\quad \forall \tau$.
\\
Per un condensatore piano, $C=\varepsilon_0 \varepsilon_r \frac{S}{d}$.


\subsection{Induttore ideale}
\begin{dfn}[Induttore ideale]
	Componente di un circuito a parametri concentrati in cui si "concentra" la proprietà dell'\textsl{induttanza} ($L$).
	\begin{figure}[H]
		\centering
		\includegraphics[width=0.8\linewidth]{induttore}
		\caption{Rappresentazione schematizzata e simbolo in un circuito a parametri concentrati di un induttore ideale}
		\label{fig:induttore}
	\end{figure}
	\[\Phi (\vec B, t) = L i(t)\]
	\[\frac{d}{dt}\Phi(\vec B, t) = L \frac{d}{dt} i(t)\]
\end{dfn}
\begin{legge}[Legge costitutiva dell'induttore ideale]
	\[v(t) = L \frac{d}{dt} i(t)\]
	\[i(t) = \frac{1}{L} \int_{-\infty}^t v(t)dt\]
	Classificazione:
	\begin{itemize}
		\item lineare
		\item controllato in corrente o in tensione
		\item con memoria (dinamico)
		\item se $i(t) = cost$, si comporta come un cortocircuito.
		\item componente passivo \[W_A(t) = L \int_{-\infty}^t i(t) \frac{di(t)}{dt} = L \int_0^{v(t)} i di = \frac{1}{2}Li^2(t) \geq 0\]
		L'energia immagazzinata nell'induttore è dovuta al campo magnetico. $i(t)$ è la variabile di stato dell'induttore.
	\end{itemize}
\end{legge}
Osserviamo la proprietà di continuità di $i(t)$.\\
Se $i(t)$ è discontinua al tempo $t= \tau$
\[i(\tau ^-) \neq i(\tau ^+)\]
\[\Rightarrow v(\tau) = L \left .\frac{d i(t)}{dt} \right |_{\tau} = \infty\]
Perciò varrebbe l'assurdo \[P_A(\tau) = v(\tau) i(\tau) = \infty \]
Se ne conclude che non è possibile avere una corrente discontinua ai capi di un induttore. Ossia $i(\tau^-) = i(\tau^+)\quad \forall \tau$.
\\